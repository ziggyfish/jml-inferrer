
\section{Conclusion}

As modern software systems grow increasingly distributed and modular, the need for tools that support long-term evolution and maintainability becomes critical. This paper presents a static analysis tool that automatically infers formal specifications from Java methods using weakest precondition and strongest postcondition analysis. By embedding these specifications into shared libraries, the tool provides a lightweight and scalable mechanism for verifying interface contracts, generating comprehensive test suites, and improving software reliability — all without requiring access to the full implementation code.

Our evaluation shows that inferred specifications significantly enhance mutation testing outcomes and test completeness across a diverse set of method categories. These improvements are particularly pronounced in categories involving control flow, factory delegation, and state modification, where traditional unit testing often falls short. By enabling automated validation of behavioral contracts at service boundaries, the tool helps detect integration bugs earlier and reduces long-term maintenance costs.

This is important because changes in shared libraries or service interfaces can easily introduce defects that manifest only during deployment. Early detection of such issues prevents costly downstream errors, improves developer confidence when modifying or extending code, and ensures that services interact correctly in complex or distributed systems. In turn, this leads to more reliable software, shorter development cycles, and reduced effort for regression testing and maintenance.

Beyond testing, the inferred specifications serve as durable, machine-checkable documentation that can guide developers during refactoring, onboarding, and evolution. This directly supports key activities in software maintenance, especially in microservice architectures where teams and services evolve independently.

In contrast to manual specification or natural-language approaches, our method requires no annotations or structured comments, making it highly suitable for industrial contexts with large, legacy codebases. The integration with mutation testing frameworks such as PiTest further demonstrates the tool’s compatibility with modern CI/CD pipelines.

While this paper focuses on specification inference and validation, future work will examine how these inferred contracts can be leveraged to improve automated testing of library clients and service integrations. This will include investigating their role in CI/CD pipelines, evaluating their effectiveness in cross-service verification, and conducting user studies to measure their impact on developer comprehension and maintenance workflows. Additional research will also explore optimizing the tool for large-scale performance, refining inference heuristics for dynamic and side-effect-heavy methods, and broadening support for other programming languages and ecosystems.

Overall, this work demonstrates that formal specification inference — when automated and integrated into development workflows and shared libraries — is both feasible and modular. By leveraging static analysis, symbolic execution, and heuristic techniques, the approach can generate specifications across large codebases without requiring manual annotation. These inferred specifications improve testability by guiding automated unit test generation, enhance reliability by detecting integration bugs and validating interface contracts, and support maintainability by providing machine-checkable documentation that captures expected behavior even as software evolves. Consequently, automated specification inference offers tangible benefits for developers working on complex, modular, or distributed systems, where ensuring correctness and reducing regression risk is particularly challenging.