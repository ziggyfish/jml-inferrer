\documentclass[10pt,conference]{IEEEtran}
\IEEEoverridecommandlockouts
% The preceding line is only needed to identify funding in the first footnote. If that is unneeded, please comment it out.


\usepackage[numbers]{natbib}
\usepackage{amsmath,amssymb,amsfonts}
\usepackage{algorithmic}
\usepackage{graphicx}
\usepackage{url} 
\usepackage[english]{babel}
\usepackage[autostyle, english = american]{csquotes}
\MakeOuterQuote{"}



\usepackage{float}
\floatstyle{plain}
\newfloat{listing}{htbp}{lop}
\floatname{listing}{Listing}

\usepackage{multirow}
\usepackage{listings}
\usepackage{xcolor} % for custom colors

\definecolor{codegray}{rgb}{0.5,0.5,0.5}
\definecolor{codeblue}{rgb}{0.0,0.0,0.6}
\definecolor{codegreen}{rgb}{0,0.5,0}

\lstdefinestyle{IEEE}{
  language=Java, % Change to your language (e.g., Java, C++)
  basicstyle=\fontsize{10}{10}\selectfont\ttfamily,
  numbers=none,
  numberstyle=\tiny\color{codegray},
  stepnumber=1,
  numbersep=5pt,
  backgroundcolor=\color{white},
  showspaces=false,
  showstringspaces=false,
  showtabs=false,
  frame=single,
  rulecolor=\color{black},
  tabsize=2,
  captionpos=b,
  breaklines=true,
  breakatwhitespace=true,
  title=\lstname,
  keywordstyle=\color{codeblue},
  commentstyle=\color{codegreen}\itshape,
  stringstyle=\color{red},
  escapeinside={(*@}{@*)}, % for LaTeX in listings
  morekeywords={*,...} % Add custom keywords if needed
}

\lstset{style=IEEE}

\usepackage{textcomp}
\usepackage{xcolor}
\def\BibTeX{{\rm B\kern-.05em{\sc i\kern-.025em b}\kern-.08em
    T\kern-.1667em\lower.7ex\hbox{E}\kern-.125emX}}

\begin{document}

\title{Enhancing Program Comprehension and Test Generation via Formal Specification Inference}

%\author{\IEEEauthorblockN{1\textsuperscript{st} Anonymous Authors}}
\author{\IEEEauthorblockN{1\textsuperscript{st} Brendan Edmonds}
\IEEEauthorblockA{\textit{UQ Cyber, University of Queensland} \\
\textit{University of Queensland)}\\
St Lucia QLD 4067, Australia \\
b.edmonds@uq.edu.au}
\and
\IEEEauthorblockN{2\textsuperscript{nd} Mark Utting}
\IEEEauthorblockA{\textit{UQ Cyber, University of Queensland} \\
\textit{University of Queensland)}\\
St Lucia QLD 4067, Australia \\
m.utting@uq.edu.au}
}


\maketitle

\begin{abstract}
Understanding and validating the behaviour of shared components in systems is critical, especially when documentation is incomplete or source code is unavailable. This paper presents a static analysis technique that automatically infers formal specifications from Java methods using weakest precondition (WP) and strongest postcondition (SP) reasoning. The approach derives precise method contracts—normal-behaviour preconditions and postconditions—and embeds them in compiled class files, with the long-term goal of facilitating improved testing of library clients.

The inferred specifications enhance program comprehension, guide automated test generation, and support long-term software maintenance, particularly in environments where APIs evolve independently. Applied to the OpenJDK core library, our tool enables AI-based unit test generation to produce up to twice as many tests as baseline methods and achieves mutation scores up to 95.7\% across method categories including accessors, factory methods, and control-flow structures. Embedding these specifications within shared libraries strengthens test completeness in safety- or compliance-critical workflows and has the potential to enable interface validation across microservices.

Overall, this work bridges formal methods and practical software engineering by showing how lightweight, inference-based specifications can improve the reliability, maintainability, and evolution of distributed systems.
\end{abstract}


\begin{IEEEkeywords}
Specification Inference, Weakest Preconditions, Program Comprehension, LLM Testing
\end{IEEEkeywords}

\section{Introduction}

Modern software systems are increasingly distributed, modular, and service-oriented. In these environments, \emph{shared libraries} play a critical role by encapsulating common functionality and standardizing communication between microservices~\cite{de_toledo_improving_2020}. However, as these systems evolve, maintaining correctness, consistency, and interoperability across components becomes significantly more complex. This complexity is exacerbated by the lack of precise behavioral specifications, particularly in large-scale or legacy codebases where documentation may be insufficient or outdated.

\emph{Formal specifications} offer a rigorous mechanism for specifying the intended behavior of software components, thereby enhancing program understanding, preventing regression, and facilitating automated verification~\cite{hasan_formal_2015, dsilva_survey_2008}. However, their adoption remains limited, as the manual development of such specifications is prohibitively expensive and error-prone, often necessitating extensive domain expertise and considerable engineering effort~\cite{matichuk_cost_2015, easterbrook_formal_1998}. Consequently, formal methods are generally constrained to high-assurance systems or isolated use cases, while the broader software industry continues to rely heavily on unit tests, which often suffer from incomplete coverage, particularly in distributed systems~\cite{pinto_inadequate_2017, zhu_software_1997}.

This paper introduces a static analysis tool that automatically infers normal-behavior specifications for existing software, providing machine-checkable contracts that improve correctness, testability, and maintainability. In this context, \textit{normal behaviour} refers to the anticipated outcomes and side effects of a program when it operates under valid inputs and within its intended environment, explicitly excluding error states or exceptional conditions. We assume that exceptions are undesirable and that well-formed client calls should be written so as not to trigger them. These inferred specifications delineate how a program should behave during standard execution, expressed as machine-checkable behavioral contracts. They enable developers to validate interface correctness, detect discrepancies in expected usage, and enhance test completeness, even when complete implementation details or manually written formal specifications are unavailable.

We demonstrate that embedding inferred specifications into shared libraries directly supports the \emph{evolution and maintenance of distributed systems} by:
\begin{itemize}
\item Enabling automated contract validation at service boundaries,
\item Enhancing regression test quality through specification-informed generation, and
\item Reducing dependency-related failures during library upgrades or refactoring.
\end{itemize}

Through a comprehensive evaluation of the OpenJDK core library, we show that our tool significantly enhances both test coverage and mutation resistance. It generates up to twice as many tests compared to baseline methods and achieves mutation scores as high as 95.7\% in several method categories, including accessors, factory methods, and control flow structures.

While this paper focuses on the inference and validation of normal-behaviour specifications, future work will examine how these inferred contracts can be leveraged to improve automated testing of library clients and service integrations. This direction aims to extend specification-based reasoning beyond individual components to the broader distributed ecosystem.

In doing so, this work contributes to a \emph{practically deployable tool} that improves software maintenance and evolution in real-world systems, bridging the gap between \emph{formal specification inference} and \emph{continuous integration workflows}~\cite{zimmermann2019lightweight}.
\section{Categorizing Method Implementations for Formal Verification}
\label{section:overview}

To evaluate our tool’s accuracy and scope, we categorize method implementations based on their structural and behavioral roles. Real-world software systems often contain a diverse mix of computational, utility, and control-flow functions, which complicates analysis if treated uniformly. Categorization clarifies this heterogeneity, allowing us to measure the tool’s performance across distinct method types rather than aggregated averages.

By grouping methods into well-defined categories, we simplify evaluation and isolate where specification inference performs best. This enables more precise assessment of both inference accuracy and applicability across diverse programming patterns. Such organization also supports comparability with prior work and clarifies how different method types respond to weakest precondition (WP) and strongest postcondition (SP) reasoning.

Our tool performs static analysis to infer formal specifications by tracing control flow and propagating correctness conditions backward through code. The analysis automatically derives:
\begin{itemize}
    \item Preconditions describing valid input states, and
    \item Postconditions defining expected output properties.
\end{itemize}
These inferred contracts act as machine-checkable documentation, improving maintainability and enabling systematic evaluation of formal verification across method categories.

\section{Weakest Precondition}

The weakest precondition method, first introduced by Dijkstra~\cite{dijkstra_guarded_1975}, identifies the essential requirements that must be satisfied before executing a program to ensure a specified postcondition holds afterward. This method systematically analyzes the program from its conclusion to its initiation, offering a structured approach to reasoning about program correctness~\cite{dijkstra1976discipline}. It facilitates the derivation of input conditions that guarantee specific outcomes.

\subsection{Formalizing Starting Conditions in Weakest Precondition Analysis}
\label{subsec:wp-starting-conditions}

In weakest precondition analysis, the \emph{starting condition} refers to the minimal requirements on the program state that must hold before executing a statement $S$ to ensure a given postcondition $Q$ holds afterward. Formally, the weakest precondition is defined as:

\[
\text{WP}(S, Q) = \{\, \sigma \mid \forall \sigma'.\, (\sigma \xrightarrow{S} \sigma') \Rightarrow Q(\sigma') \,\}
\]

That is, all initial states $\sigma$ for which every possible outcome $\sigma'$ of executing $S$ satisfies $Q$.

For example, given the statement $x := x + 1$ and postcondition $Q = (x > 5)$, the weakest precondition is:

\[
\text{WP}(x := x + 1, x > 5) = x > 4
\]

This expresses that the input state must satisfy $x > 4$ to guarantee that $x > 5$ holds after execution.

In practice, weakest precondition reasoning proceeds backward through code: for a sequence $S_1; S_2$, we compute $\text{WP}(S_1; S_2, Q) = \text{WP}(S_1, \text{WP}(S_2, Q))$. This compositionality enables precise precondition inference from a desired outcome, supporting formal verification, test generation, and behavioral specification inference.


\subsection{Assignment Statements}
\label{sec:assignment-statements}
For assignment statements, the weakest precondition is determined by substituting every occurrence of the assigned variable in the postcondition with the corresponding right-hand side expression~\cite{dijkstra1976discipline}. However, this substitution must also ensure that the right-hand side expression is \emph{well-defined}. This means that any potential runtime errors—such as division by zero, null dereferences, or out-of-bounds array accesses—must be excluded.

Formally, for an assignment $x := e$ and postcondition $Q$, the weakest precondition is expressed as:
$$
\text{WP}(\, x := e, Q) = \text{def}(e) \land Q[x \leftarrow e] 
$$
where $\text{def}(e)$ \cite{morgan_programming_1990} represents the conditions under which the expression $e$ is well-defined (e.g., array indices are within bounds and divisors are non-zero). The conjunction ensures that both the expression $e$ is safe to evaluate and the postcondition $Q$ holds after the substitution.

For example, given $Q = (x + y > 10)$ and $x := y + 2$, we compute:
$$
\text{WP}(\, x := y + 2, Q) = ((y + 2) + y > 10) = (2y + 2 > 10)
$$

\subsection{Conditional Statements}
Conditionals require the postcondition to hold across all branches. For a conditional statement with condition $C$, true branch $S_1$, and false branch $S_2$, the weakest precondition is:
$$
\text{WP}(\text{if } C, Q) = (C \rightarrow \text{WP}(S_1, Q)) \land (\neg C \rightarrow \text{WP}(S_2, Q))
$$
Given the example:
\begin{verbatim}
if (x > 5)
  y := x + 1
else
  y := x - 1
\end{verbatim}
and $Q = (y > 6)$, we derive:
$$
\begin{aligned}
& (x > 5 \rightarrow \text{WP}(y := x + 1, y > 6)) \land \\
& (\neg(x > 5) \rightarrow \text{WP}(y := x - 1, y > 6))
\end{aligned}
$$

Computing the assignment preconditions:
$$
(x > 5 \rightarrow x + 1 > 6) \land (x \leq 5 \rightarrow x - 1 > 6)
$$

This can be simplified to:
$$
(x > 5 \rightarrow x > 5) \land (x \leq 5 \rightarrow x > 7)
$$

\subsection{Return Statements}

Since the weakest precondition (WP) calculus does not treat \texttt{return} as a primitive, we model it by assigning a distinguished variable \texttt{result} followed by a termination operation~\cite{morgan_programming_1990}. Thus, \texttt{return e} is modeled as \texttt{result := e}, and we derive its weakest precondition using the standard assignment rule (see Section~\ref{sec:assignment-statements}). Moreover, as the return statements are the final statements executed in the control flow, they serve as the starting point from which all WPs are calculated. The initial postcondition will be true, reflecting that we infer only \emph{normal behaviour} specifications. Exceptional behaviour, such as that arising from \texttt{throw} statements, lies outside the scope of these specifications and is instead treated as a violation of the contract.


\subsection{Throw Statements}

In contrast to \texttt{return}, the weakest precondition (WP) calculus does not provide a direct primitive for modeling \texttt{throw} statements. A \texttt{throw} disrupts the normal control flow by transferring execution to an exception handler, which lies outside the scope of the normal behaviour specifications considered in this work. Since we assume that exceptions are undesirable and client code should avoid triggering them, we model \texttt{throw} as an immediate violation of the specification. 

Formally, we equate the WP of a \texttt{throw} statement with \texttt{false}, reflecting that no subsequent program state can satisfy the intended contract once an exception is raised. This captures the intuition that a program containing a reachable \texttt{throw} is inconsistent with the assumption of exception-free normal execution. 

\paragraph{Example.}  
Consider a statement of the form:
\[
    \texttt{throw e;}
\]
For any postcondition \(Q\), the weakest precondition is given by:
\[
    WP(\texttt{throw e}, Q) = \texttt{false}.
\]
This means that the presence of a \texttt{throw} statement generates a weakest precondition that requires all clients to avoid execution paths leading to the exception. In other words, the weakest precondition enforces that program states must prevent the \texttt{throw} from being reached, as no postcondition \(Q\) can be guaranteed once an exception is raised.


\subsubsection{Summary of Weakest Preconditions}
By iteratively applying these transformations to each line of code in reverse order, the tool calculates the weakest precondition necessary for the overall postcondition to hold. This backward propagation serves as the foundation of the analysis and facilitates the inference of precise specifications for the function.


\section{Strongest Postconditions and Loop Inference}
\label{section:strongest-postconditions}

In complement to weakest precondition (WP) analysis, our study explores the inference of strongest postconditions (SP) to enrich formal specification generation. While WP propagates correctness constraints backward from a postcondition, SP computes the forward effects of program execution, starting from known initial conditions. This directionality offers distinct advantages in dynamic verification, symbolic execution, and abstract interpretation.

Inferring strongest postconditions is particularly valuable when input conditions are fully known or can be assumed from the program context. This technique allows precise modeling of the exact outcomes and state transitions induced by a sequence of statements, thus aiding in specification discovery, test generation, and formal verification. It also enables automatic derivation of postconditions from implementations—a common challenge when documentation is sparse or outdated.

Incorporating SP inference into our categorization framework further delineates method complexity, especially in the presence of loops, mutable state, or compound branching. SP-based analysis becomes essential for verifying cumulative behavior in these scenarios and for refining inferred contracts during program synthesis or test oracle generation.

\subsection{Formalizing Starting Conditions in Strongest Postcondition Analysis}
\label{subsec:sp-starting-conditions}

Strongest postcondition (SP) analysis begins from a known initial condition, typically a logical predicate over program variables. In our case, this initial condition may originate from the precondition specified by the developer using the Java Modeling Language (JML) or could be derived from the previous weakest precondition (WP) analysis. These JML annotations delineate the assumptions that must be satisfied upon method entry.

Given a program statement $S$ and a JML-specified precondition $P$, the strongest postcondition $\text{SP}(P, S)$ describes all possible program states that may result from executing $S$ starting in any state that satisfies $P$.

Formally:
\[
\text{SP}(S, P) = \{\, \sigma' \mid \exists \sigma.\, P(\sigma) \land \sigma \xrightarrow{S} \sigma' \,\}
\]

Here, $\sigma$ is an initial program state that satisfies the JML precondition $P$, and $\sigma'$ is the resulting state after executing $S$. The relation $\sigma \xrightarrow{S} \sigma'$ denotes that executing $S$ in state $\sigma$ produces state $\sigma'$.

In practice, JML preconditions act as the entry-point assumptions for strongest postcondition (SP) inference. They are carried through the analysis to record the conditions that must hold after each possible execution path.

For loops, this starting condition may either be a loop invariant (if one is provided) or the program state at the beginning of loop unrolling. During SP-based reasoning, the initial predicate 

P is updated across successive iterations, converging toward an approximation of the loop’s cumulative effect. This process corresponds to computing the least fixpoint of the loop semantics, as described in Section~\ref{section:strongest-postconditions}.


\subsection{Strongest Postcondition}

The strongest postcondition, denoted $\text{SP}(S, P)$ for precondition $P$ and statement $S$, represents the most precise description of the program state after executing $S$, assuming the state initially satisfies $P$~\cite{morgan_programming_1990, hoare1969axiomatic}. Unlike WP, which checks that a postcondition is guaranteed after execution, SP enumerates what \emph{actually holds} after execution from a given starting state.

When a method accepts parameters, each is renamed to preserve its initial value during strongest postcondition (SP) analysis. For example:

\[
\texttt{int foo(T x\_in)\{T x=x\_in; S; return x;\}}
\]
Here, \texttt{x\_in} records the parameter’s entry value, allowing assignments such as \texttt{x = x + 1} to be expressed as \texttt{x := x\_in + 1}. The return statement then yields \texttt{result := x\_in + 1}, making the transformation from input to output explicit.


\subsubsection{Assignment Statements}

For an assignment $x := e$, and a precondition $P$, the SP is obtained by existentially quantifying over the variable being assigned and expressing that it assumes the new value:

\[
\text{SP}(\, x := e, P) = \exists x_0.\, (P[x \leftarrow x_0] \land x = e[x \leftarrow x_0])
\]

\textbf{Example:} Let $P = (x > 5)$ and $x := x + 1$. Then:
\[
\text{SP}(\, x := x + 1, x > 5) = \exists x_0.\, (x_0 > 5 \land x = x_0 + 1)
\]
which simplifies to:
\[
x > 6
\]

\subsubsection{Conditional Statements}

For a conditional statement of the form:

\begin{verbatim}
if (C) then S1 else S2
\end{verbatim}

the strongest postcondition is defined as a disjunction over both branches:

\[
\text{SP}(\text{if } C \text{ then } S_1 \text{ else } S_2, P) = 
\text{SP}(S_1, P \land C) \lor \text{SP}(S_2, P \land \neg C)
\]

\subsubsection{Return Statements}

As in WP, return statements are translated to assignments to a special variable \texttt{result}. SP propagates forward from the assignment:

\begin{align*}
\text{SP}&(\, \texttt{return}\ e, P) 
  = \text{SP}(\, \texttt{result} := e, P) \notag \\
  &= \exists r_0.\, \big(P[\texttt{result} \leftarrow r_0] 
     \land \texttt{result} = e[\texttt{result} \leftarrow r_0] \big)
\end{align*}

\textbf{Example:} Let $P = (x > 0)$ and \texttt{return x + 1}. Then:
\[
\text{SP}(\texttt{return } x + 1, x > 0) = x > 0 \land \texttt{result} = x + 1
\]

\subsection{Loop Inference with Strongest Postconditions}
\label{section:loops-sp}

Loop inference poses a significant challenge in formal verification due to the need to reason about potentially unbounded iteration and the need for invariants. In the strongest postcondition (SP) analysis, the goal is to characterise the set of all possible states resulting from executing a loop, starting from a given precondition $P$.

To achieve this, we iteratively apply SP over the loop body until reaching a fixpoint that captures the cumulative effect of all iterations. This is formally expressed as:

\[
\text{SP}(\, \text{while } C \text{ do } S, P) = \mu X.\, \left(P \lor \text{SP}(S, X \land C)\right)
\]

where $\mu X$ represents the least fixpoint—a set of states closed under repeated application of the loop body $S$ while condition $C$ holds.

In practice, exact fixpoint computation is undecidable for general loops. Therefore, our tool employs heuristic techniques to infer loop behavior:

\begin{itemize}
    \item \textbf{Invariant Guessing:} Derived from syntactic patterns, such as monotonic counter updates.
    \item \textbf{Symbolic Unrolling:} Executes the loop body a bounded number of times to detect emergent patterns.
    \item \textbf{Widening Operators:} Used to generalize from partial traces and stabilize convergence.
\end{itemize}

\textbf{Example:}
\begin{verbatim}
int sum = 0;
for (int i = 0; i < n; i++) {
    sum += i;
}
\end{verbatim}

\paragraph{Deriving the postcondition}  
The tool analyzes the loop as follows:  
1. \textbf{Initial state:} \( \texttt{sum} = 0, i = 0 \).  
2. \textbf{Symbolic unrolling:} The loop body is executed symbolically for the first few iterations:  
   \[
   \begin{aligned}
   i = 0 &\Rightarrow \texttt{sum} = 0\\
   i = 1 &\Rightarrow \texttt{sum} = 0 + 0 = 0\\
   i = 2 &\Rightarrow \texttt{sum} = 0 + 1 = 1\\
   i = 3 &\Rightarrow \texttt{sum} = 1 + 2 = 3
   \end{aligned}
   \]  
   From this pattern, the tool infers an arithmetic accumulation.  
3. \textbf{Invariant generalization / widening:} Using the detected pattern, the loop invariant is approximated as  
   \[
   \texttt{sum} = \sum_{k=0}^{i-1} k
   \]  
   which, at loop termination (\(i = n\)), yields the postcondition:  
   \[
   \texttt{sum} = \sum_{i=0}^{n-1} i = \frac{n(n-1)}{2}.
   \]

\paragraph{Limitations of heuristics}  
If the heuristics fail to find a suitable loop invariant (e.g., for non-linear updates, complex branching, or data-dependent loops), the inferred postcondition may be weaker or overly conservative. In such cases:  

\begin{itemize}
    \item The inferred SP may only capture partial behavior, such as upper/lower bounds instead of exact values.  
    \item Final method specifications remain sound but less precise, meaning they guarantee correctness for a subset of normal behaviors but may not express the full effect.  
    \item In practice, this occurs rarely for loops with simple arithmetic or monotonic updates, but more often for loops manipulating complex data structures or with nested loops.
\end{itemize}

\paragraph{Impact of weaker specifications}  
Weaker postconditions do not invalidate the overall correctness of the tool; they still allow clients to reason about normal behaviour safely. However, overly conservative specifications may reduce the utility of inferred contracts for test generation or interface validation, potentially requiring manual refinement for critical methods. In general, having some methods with weaker specifications is acceptable, as most normal program behavior is still captured accurately.

\section{Shared Libraries and Weakest Preconditions}
Shared libraries are essential components in distributed systems as they provide reusable methods that standardize communication protocols, including network requests, data serialization, and inter-service messaging. By abstracting these complexities, shared libraries enhance consistency and minimize redundancy across the system. This section elucidates common categories of method implementations, with particular emphasis on weakest precondition analysis, utilizing Java examples.

Our goal is to systematically categorize these methods and infer their specifications. Weakest preconditions provide a formal mechanism for verifying that certain conditions must hold for a method's execution to satisfy its postcondition.

\label{section:methodImplementation}

\subsection{Accessors \& Mutators}
Accessors (getters) and mutators (setters) are methods utilized to encapsulate and access class properties. Despite their seemingly straightforward nature, frequent use of these methods can significantly impact testing metrics~\cite{fowler2004refactoring}.

\begin{listing}
\begin{lstlisting}[language=Java, caption={Example of Accessor and Mutator}]
public class User {
    private String name;

    public String getName() {
        return name;
    }

    public void setName(String name) {
        this.name = name;
    }
}
\end{lstlisting}
\end{listing}

For accessors and mutators, the weakest precondition (WP) and strongest postcondition (SP) are relatively straightforward.

\begin{itemize}
    \item \textbf{Accessor (getter):} The WP is trivially \texttt{true}, as no state changes occur. The SP is that the returned value equals the current field value, i.e., \texttt{result = name}.
    \item \textbf{Mutator (setter):} The WP depends on input constraints specified by the specifications. For example, \texttt{name != null \&\& name.length() > 0} if non-empty input is required. The SP is that the field is updated: \texttt{this.name == name}.
\end{itemize}


\subsection{Control Flow Structures}
This category encompasses conditional logic and exception handling. These constructs introduce branches that are essential for evaluating test path coverage and mutation testing~\cite{ammann2016introduction}.

\begin{listing}
\begin{lstlisting}[language=Java, caption={Conditional Logic Example}]
public int calculateDiscount(int quantity) {
    if (quantity > 10) {
        return 20;
    } else {
        return 10;
    }
}
\end{lstlisting}
\end{listing}

For control flow structures, the weakest precondition (WP) determines the necessary conditions for correct branching, while the strongest postcondition (SP) describes the expected outcomes associated with each branch.

\begin{itemize}
    \item \textbf{Conditional Logic:}  
    - \textbf{WP:} Depends on the condition and desired postcondition. For the \texttt{if} statement above, the WP ensures that the postcondition holds after either branch:  
   \begin{align*}
WP(&\texttt{if (quantity > 10)} \\
  &\texttt{then result := 20 else result := 10}, Q) \\
&= ((quantity > 10) \Rightarrow WP(result := 20, Q)) \\
&\quad \wedge ((quantity \le 10) \Rightarrow WP(result := 10, Q))
\end{align*}
    - \textbf{SP:} Represents the actual outcome given a starting state. For example, if \texttt{quantity = 15}, \(\texttt{SP} = (result = 20)\); if \texttt{quantity = 5}, \(\texttt{SP} = (result = 10)\).
\end{itemize}

\subsection{Factory \& Delegate Patterns}
Factory methods encapsulate object creation, while delegate methods forward calls. These patterns promote modularity and reuse; however, they necessitate precise testing due to their indirect behavior~\cite{bloch2008effective}.
\begin{listing}
\begin{lstlisting}[language=Java, caption={Factory and Delegate Methods}]
public static Connection createConnection() {
    return new DatabaseConnection();
}

public void save(User user) {
    database.save(user);
}
\end{lstlisting}
\end{listing}
For the factory and delegate patterns, the weakest precondition (WP) delineates the necessary conditions for object creation or method forwarding. Conversely, the strongest postcondition (SP) characterizes the expected outcome of the method's execution.

\begin{itemize}
    \item \textbf{Factory Method:} The WP for a factory method ensures that the required parameters are provided, if any. For example, \texttt{createConnection()} requires no parameters, so the WP is trivially true. The SP is the object created, in this case, a \texttt{DatabaseConnection}.
    \item \textbf{Delegate Method:} The WP for a delegate method ensures the input is valid (e.g., \texttt{user != null} for the \texttt{save} method). The SP is the result of forwarding the call to the delegate, in this case, a \texttt{save} action performed on the \texttt{user} object.
\end{itemize}

\subsection{Utility methods}
often defined as static, perform essential helper operations such as formatting and calculation. While their logic is straightforward, they are vital to the overall functionality~\cite{martin2008clean}.
\begin{listing}
\begin{lstlisting}[language=Java, caption={Utility and Intermediate Variable Methods}]
public static String capitalize(String input) {
    return input.substring(0, 1).toUpperCase() + input.substring(1);
}

public int doubleValue(int x) {
    int result = x * 2;
    return result;
}
\end{lstlisting}
\end{listing}
For utility methods, the weakest precondition (WP) establishes the necessary conditions that must be satisfied for the operation to be valid. In contrast, the strongest postcondition (SP) articulates the expected outcome of the method's execution.

\begin{itemize}
    \item \textbf{Utility Methods:} The WP for utility methods involves the input being valid for the operation (e.g., non-null string for a string manipulation method). The SP is the result of the operation, such as the transformation applied to the string.
    \item \textbf{Intermediate Variable Methods:} The WP ensures that the method’s input is valid (e.g., non-zero if the input is used as a divisor). The SP is the result of the computation, typically a value computed using the input. 
\end{itemize}

\subsection{Other Methods}
This category encompasses various logical frameworks and event-handling techniques. These techniques are often employed indirectly and require tailored testing strategies~\cite{gamma1994design}.
\begin{listing}
\begin{lstlisting}[language=Java, caption={Event Handler / Miscellaneous Method}]
public void handleClick(Event e) {
    System.out.println("Button clicked!");
}
\end{lstlisting}
\end{listing}

For methods such as event handlers and other miscellaneous methods, the weakest precondition (WP) typically involves verifying the validity of necessary parameters, such as the event object. The strongest postcondition (SP) generally specifies the expected side effects, which may include a printed message or a change in state.

\begin{itemize}
    \item \textbf{Other Methods:} The WP for these methods ensures that required inputs, like event data, are valid. The SP includes the expected side effects, such as logging, UI updates, or other actions performed by the method.
\end{itemize}

\subsection{State Modification}
State modification methods directly alter the internal state of objects. These methods differ from standard mutators by incorporating additional business logic, making them essential for behavioral testing~\cite{pressman2014software}.
\begin{listing}
\begin{lstlisting}[language=Java, caption={State-Changing Method}]
public void applyDiscount(double percentage) {
    if (percentage > 0) {
        this.price = this.price - (this.price * percentage);
    }
}
\end{lstlisting}
\end{listing}


\section{Results}
\label{section:results}

In our study, we evaluated the effectiveness of specification inference methods by comparing the number of additional unit tests generated with and without the specifications inferred by our approach across various function categories. The results demonstrate a significant improvement in the generation of unit tests when inferred specifications are applied.

Specifically, across all evaluated categories, the application of inferred specifications resulted in a significant increase in the number of methods identified for unit testing. In particular, categories such as Factory \& Delegate Patterns and Accessors \& Mutators exhibited the highest improvement in test generation, achieving mutation rates of 95.77\% and 94.25\%, respectively, when specifications were inferred. Furthermore, the Control Flow Structures category, which encompasses conditional logic and looping constructs, demonstrated a mutation rate of 94.25\% with inferred specifications, underscoring the substantial impact of formal specifications on the overall quality of test generation.

The Other Method category also saw a substantial improvement in test generation, achieving a mutation rate of 83.39\% when inferred specifications were applied. This further underscores the utility of specifications in less conventional methods.

These findings reveal a significant increase in the number of methods identified for unit testing, resulting in a total increase of 76\%. This underscores the potential of formal specification inference in enhancing test generation and improving software quality across a wide range of function categories. For detailed results, refer to Table 1.

\begin{table*}[t]
\centering
\begin{tabular}{|l|r|r|r|r|r|r|r|}
\hline
\multirow{2}{*}{\textbf{Combined Category}} & 
\multirow{2}{*}{\textbf{Methods}} & 
\multicolumn{2}{c|}{\textbf{Sig-only (P1)}} & 
\multicolumn{2}{c|}{\textbf{Sig + Guide (P2)}} &
\multicolumn{2}{c|}{\textbf{Sig + Spec (P3)}} \\
\cline{3-8}
 & & \textbf{Tests} & \textbf{Mut. (\%)} & \textbf{Tests} & \textbf{Mut. (\%)} & \textbf{Tests} & \textbf{Mut. (\%)} \\
\hline
Accessors \& Mutators        & 2,889 & 13,358 & 75.19 & 17,234 & 70.42 & 22,537 & 94.25 \\
\hline
Abstract Constructs          & 509   & 2,565  & 92.88 & 3,147  & 68.55 & 4,494  & 89.83 \\
\hline
Control Flow Structures      & 930   & 10,310 & 47.17 & 12,134 & 62.18 & 19,628 & 94.25 \\
\hline
Factory \& Delegate Patterns & 4,773 & 28,941 & 76.28 & 35,276 & 69.87 & 48,104 & 95.77 \\
\hline
Support \& Utility Methods   & 355   & 2,838  & 52.70 & 3,053  & 63.45 & 4,878  & 93.26 \\
\hline
Other Method                 & 751   & 4,311  & 48.96 & 5,118  & 60.12 & 7,916  & 83.39 \\
\hline
State Modification           & 502   & 2,664  & 56.94 & 3,612  & 63.25 & 5,124  & 93.47 \\
\hline
\textbf{Average}             &       &        & \textbf{57.59} &   & \textbf{68.94} &        & \textbf{91.65} \\
\hline
\end{tabular}
\caption{Mutation and test metrics by method type for OpenJDK, corresponding to the three experimental phases: P1 (signature-only), P2 (signature + guidance), and P3 (signature + formal specification).}
\label{tab:mutation-method-summary}
\end{table*}

\subsection{Inference of Function Specifications}

To assess the impact of formal specification on testing outcomes, we conducted an experiment using functions from the OpenJDK core library. The experiment had three phases. 

\textbf{Phase 1: Signature-only tests.} Google's Gemini 2.0 AI generated unit tests based solely on the method signature, simulating a typical developer without formal specification guidance. These tests covered common and some edge cases but were not exhaustive.


\begin{lstlisting}[frame=none,language=TeX] 
Prompt 1: Write unit tests for the following function signature under typical development constraints. Provide the tests as a developer would normally write them, without any formal specification guidance:
\end{lstlisting}
\vspace{-5ex}
\begin{lstlisting}[caption={Example method signature for computing the midpoint between two geographic coordinates.}]
GeographicalPoint midpoint(float lat1, float lon1, float lat2, float lon2);
\end{lstlisting}
\label{lst:midpoint-signature}


\textbf{Phase 2: Signature + descriptive guidance.} The AI was instructed to generate a comprehensive set of unit tests using only the function signature, but explicitly told to cover normal behavior, edge cases, and potential failure scenarios:

\begin{lstlisting}[frame=none]    
Prompt 2: Given the following function signature, write a comprehensive set of unit tests. Ensure that the tests cover normal behavior, edge cases, and potential failure scenarios:
\end{lstlisting}
\vspace{-4ex}

\textbf{Phase 3: Signature + formal specification.} The AI received both the function signature and a formal specification, including precise preconditions and postconditions. This enabled the generation of fully comprehensive tests covering all expected behaviors and edge cases:

\begin{lstlisting}[frame=none]  
Prompt 3: Given the following function signature and formal specification, write a comprehensive set of unit tests. Ensure that the tests cover normal behavior, edge cases, and potential failure scenarios:
\end{lstlisting}
\vspace{-4ex}

\begin{listing}
\begin{lstlisting}[label={lst:midpoint-spec},caption={Specification of the \texttt{midpoint} function, defining preconditions for valid latitude and longitude ranges and postconditions for the computed midpoint.}]
/* 
  @requires -90 <= lat1 && lat1 <= 90 &&
    -180 <= lon1 && lon1 <= 180 &&
    -90 <= lat2 && lat2 <= 90 &&
    -180 <= lon2 && lon2 <= 180;
  @ensures \result != null &&
    \result.lat == (lat1 + lat2) / 2 &&
    \result.lon == (lon1 + lon2) / 2;
*/

GeographicalPoint midpoint(float lat1, float lon1, float lat2, float lon2);
\end{lstlisting}
\end{listing}

Comparing the three phases, we observed a clear progression in coverage and detail: signature-only tests provided minimal coverage, signature-plus-guidance tests increased breadth, and formal specification produced the most comprehensive and reliable test suite. This demonstrates the value of both descriptive guidance and formal specifications in improving software reliability and reducing bugs.


\subsection{Mutation Testing with PiTest}

To further validate the impact of inferred specifications on test robustness, we employed \textit{PiTest}, a widely used mutation testing framework for Java \cite{coles2016pitest}. Mutation testing introduces small faults (called \textit{mutants}) into the code and evaluates whether the test suite can detect these faults. A high mutation score reflects the effectiveness of a test suite in identifying latent defects \cite{ammann2016introduction}.

\subsubsection*{Experimental Process}

We applied PiTest to all function categories in two stages:
\begin{enumerate}
    \item \textbf{Without inferred specifications} – using unit tests created through conventional test generation (as described in Section 3.1).
    \item \textbf{With inferred specifications} – using unit tests derived from formal specifications.
\end{enumerate}

PiTest was configured to apply standard mutation operators, which include:
\begin{itemize}
    \item Replacing arithmetic operations (e.g., \texttt{+} $\rightarrow$ \texttt{-})
    \item Altering conditional logic (e.g., \texttt{>} $\rightarrow$ \texttt{>=})
    \item Removing or mutating return statements
\end{itemize}

\subsubsection*{Example: \texttt{midpoint()} Function}

Using the formal specification defined in Listing~\ref{lst:midpoint-spec}, we evaluated the effectiveness of specification-informed tests in identifying faults that were introduced through mutation.

\paragraph{Mutation: Arithmetic Replacement}

A representative mutation introduced by PiTest altered the arithmetic logic of the function in the following manner:

\begin{verbatim}
// Original
(lat1 + lat2) / 2

// Mutated
(lat1 - lat2) / 2
\end{verbatim}

This mutation distorts the expected midpoint calculation, particularly in cases with asymmetric inputs, where the latitudes differ significantly in sign.

\paragraph{Test Suite Effectiveness}

\begin{itemize}
    \item \textbf{Without specification-based tests}, this mutant survived. The baseline suite lacked a test that would detect incorrect behavior when inputs involved differing signs or directional extremes.
    \item \textbf{With specification-based tests}, the following test case successfully detected and killed the mutant:
\end{itemize}

\begin{listing}
\begin{lstlisting}[caption={Unit test verifying the \texttt{midpoint} function correctly computes results for coordinates with mixed signs.}]
@Test
void testMixedSigns() {
    GeographicalPoint result = midpoint(-10, 20, 30, -40);
    assertNotNull(result);
    assertEquals(10.0, result.lat);
    assertEquals(-10.0, result.lon);
}
\end{lstlisting}

\end{listing}

In this test, inputs \texttt{lat1 = -10} and \texttt{lat2 = 30} yield an expected latitude of \texttt{10.0}. The mutated version would compute \texttt{(-10 - 30)/2 = -20.0}, which clearly violates the specification.

\subsubsection*{Summary of Results}

Across the evaluated functions, test suites generated using inferred specifications demonstrated:

\begin{itemize}
    \item Significantly higher mutation coverage in complex function categories (e.g., control-flow heavy methods, factory methods, and delegation patterns)
    \item Greater fault detection for edge-case and off-nominal mutants that often eluded baseline developer-style tests
    \item Stronger alignment with functional contracts, enabling precise validation of arithmetic and state-based correctness
\end{itemize}

The integration of PiTest into our workflow confirmed that inferred specifications guided the generation of more comprehensive test cases and enhanced fault tolerance. These specification-informed tests focused on semantic correctness, effectively bridging the gap between developer intent and formal assurance.

\section{Discussion}

The results demonstrate that automated specification inference is a powerful tool for supporting software evolution and maintenance, particularly in distributed systems. By embedding machine-verifiable contracts into shared libraries, our tool provides clear benefits to developers working across microservices, modular architectures, and legacy systems. These contracts act as executable specifications automatically checked during development and integration. When a developer updates a microservice, shared library contracts immediately highlight violations of expected behavior, preventing faulty deployments. In modular systems, they ensure that changes in one component do not break dependent interfaces. In legacy systems, contracts provide a safety net by formally specifying expected behavior, enabling automated test generation and fault detection even when documentation is incomplete. Overall, developers can use these contracts to guide coding, validate integration, and maintain correctness throughout the software lifecycle.

A key finding is the improvement in mutation coverage and unit test completeness across diverse method categories. \emph{Control Flow Structures}, \emph{Factory \& Delegate Patterns}, and \emph{Accessors \& Mutators} achieved mutation coverage exceeding 94\%, showing that inferred specifications effectively verify correctness across different behavior types — a crucial factor in maintaining reliable software.

Unlike prior work relying on manually written specifications~\cite{hasan_formal_2015, andronick_large-scale_2012} or natural language documentation~\cite{toradocu2016, jdoctor2018}, our method provides an automated, scalable solution integrated directly into modern toolchains. The inferred specifications are derived from program logic, ensuring consistency with implementation and mitigating documentation drift.

Beyond improved testing, the tool contributes to the \textit{evolutionary lifecycle} of distributed software. As systems change and teams rotate, maintaining confidence in shared behavior remains a challenge. Our approach addresses this by generating \emph{contractual documentation} embedded within reusable libraries, enabling validation even when source code is unavailable — a common scenario in microservice architectures.

These findings support the view that formal methods can deliver practical value when automated appropriately. By reducing the effort to generate specifications and integrating them into CI/CD pipelines, our tool bridges the gap between formal verification and agile development~\cite{matichuk_cost_2015, martin2008clean}.

Several limitations warrant discussion. While the approach achieves strong results across diverse function types, its effectiveness is lower for highly dynamic or side-effect-heavy methods, such as event-driven handlers or complex I/O. The computational cost of weakest precondition analysis also grows with control-flow complexity. Although acceptable in our evaluation, optimizing inference for industrial-scale systems remains an important direction for future work.

Finally, while we demonstrated test generation and mutation resilience as measurable benefits, additional developer studies (e.g., comprehension or onboarding tasks) would help validate the real-world utility of inferred specifications from a human-factors perspective~\cite{gamma1994design, fowler2004refactoring}.

In summary, the proposed tool improves both test robustness and long-term maintainability. It provides a foundation for research into specification-based software evolution and represents a step toward scalable formal reasoning tools that integrate naturally into modern engineering workflows.

\section{Related Works}

Formal specification inference has been a prominent area of research in software verification. Numerous studies focus on the automated generation of specifications to enhance software reliability and testing. Several approaches exist for inferring specifications, including:

One of the foundational approaches to formal verification is theorem proving and model checking. Hasan and Tahar \cite{hasan_formal_2015} provide an overview of formal verification techniques, highlighting the role of theorem provers in ensuring software correctness. However, theorem proving traditionally requires manually written formal specifications, which can be costly and time-consuming. Similarly, the work by Andronick et al. \cite{andronick_large-scale_2012} describes large-scale formal verification in industry settings, illustrating the challenges associated with manually crafting specifications along with the high costs involved in the verification process.

D'Silva et al. \cite{dsilva_survey_2008} provide a comprehensive survey of automated formal verification techniques. They note that while static analysis methods, including weakest precondition propagation and symbolic execution, have been employed for specification inference, these techniques often encounter challenges related to scalability and the management of complex real-world codebases. Easterbrook and Callahan \cite{easterbrook_formal_1998} discuss the validation of partial specifications, which is relevant to our study. However, their approach does not leverage automatic inference techniques, rendering it less adaptable to contemporary software development environments.

A more recent development in specification inference is the use of shared libraries in distributed systems to standardize communication and ensure data integrity. De Toledo et al. \cite{de_toledo_improving_2020} explore how microservices leverage shared libraries to maintain consistency across services. Our study builds upon their findings by integrating inferred specifications into shared libraries. This integration enhances the reliability of these libraries and supports automated verification within microservice architectures. This work extends their findings by embedding inferred specifications into shared libraries, thereby enhancing their reliability and supporting automated verification in microservice architectures.

Matichuk et al. \cite{matichuk_cost_2015} conducted an empirical study on the cost-effectiveness of formal verification, concluding that the manual development of specifications remains expensive despite the availability of theorem provers. Our study addresses this cost issue by automating specification inference, thereby reducing reliance on manual intervention.

Another key issue in software testing is the adequacy of unit tests. Zhu et al. \cite{zhu_software_1997} and Pinto et al. \cite{pinto_inadequate_2017} highlight the limitations of unit testing, which stem from insufficient test coverage and the overconfidence of developers in existing tests. Our study builds upon this by demonstrating that inferred specifications significantly enhance test coverage and robustness, particularly in distributed systems where traditional unit testing poses challenges.

\subsection{Complementary Approaches: Toradocu/Jdoctor and PreInfer}

The Toradocu tool~\cite{toradocu2016}, along with its enhanced version Jdoctor~\cite{jdoctor2018}, exemplifies a notable application of natural language processing (NLP) to extract exception-related preconditions from Java code. These tools interpret structured natural language within Javadoc comments to infer preconditions, which are expressed as Java Boolean conditions. Such inferred conditions can be used to automatically generate test oracles, guide unit test generation, or enrich formal specifications, ensuring that software behaves as documented. In evaluations using popular Java libraries, Jdoctor demonstrated a recall of 83\% and a precision of 92\%, underscoring its effectiveness in recovering documented behavior that can be directly leveraged for testing and verification.

Our approach and Toradocu/Jdoctor serve complementary roles:

\begin{itemize}
    \item Toradocu and Jdoctor rely on the presence of structured documentation, enabling them to extract preconditions effectively when they are described. However, this dependence on natural language also introduces limitations in terms of precision, as natural language can be inherently ambiguous.
    
    \item In contrast, our method infers specifications by analyzing the actual program logic through static techniques. This allows for high precision --- inferred conditions are correct by construction --- though it may lead to lower recall when some behaviors are not explicitly represented in the control flow. This trade-off makes our method especially suited to environments where correctness is critical, such as distributed or microservice-based systems.
\end{itemize}

Importantly, only a subset of the exception preconditions discovered by our technique is actually present in Javadoc comments. For example, consider a method that parses user input and throws a \texttt{NumberFormatException} when the input is invalid. This exception may not be documented in the Javadoc, so a tool like Jdoctor would not recover it. In contrast, our method can automatically infer this precondition from the code, revealing that the input must satisfy a numeric format constraint. This underscores a limitation of comment-based approaches like Jdoctor: they can only recover what developers have chosen to document. By uncovering latent specifications, our method provides a more complete understanding of exceptional behavior.

PreInfer \cite{preinfer2017} is another relevant contribution, focusing on the C\# language. It infers preconditions by analyzing failing executions using symbolic execution, specifically via the Pex testing framework. PreInfer is capable of synthesizing intricate conditions, including disjunctions and quantified properties over arrays. In contrast to our work, PreInfer occupies a different region of the design space:

\begin{itemize}
    \item PreInfer is optimized for identifying complex conditions in computationally intensive functions, often found in data structure and algorithm implementations. While Pex systematically explores execution paths, the sheer number of possible paths in complex or recursive functions can make exhaustive analysis infeasible. As a result, some paths may not be explored within practical time or resource limits, and the inferred conditions may not be guaranteed to be complete or sound.
    
    \item Our technique is engineered for scalability and precision, aiming to infer simpler, highly reliable preconditions. This focus makes it more applicable to large-scale software systems, where individual units may be less complex but system-wide reliability is crucial.
\end{itemize}

\subsection{Extending the Capacity of Existing Work}

Our work extends prior specification inference tools such as Toradocu/Jdoctor and PreInfer by focusing on automatic inference of formal specifications through weakest precondition (WP) analysis. While Toradocu and Jdoctor extract exception preconditions from Javadoc comments, their effectiveness is limited by the availability and clarity of documentation. In contrast, our approach infers precise specifications directly from source code, making it suitable for large, real-world systems where documentation is sparse or outdated.

PreInfer infers complex preconditions in C\# using symbolic execution via Pex but suffers from incomplete coverage due to runtime path exploration. Our WP-based static analysis instead emphasizes soundness and scalability, tailored for Java environments—particularly distributed and microservice systems—where consistent inter-service contracts are critical.

A key contribution of our work is integrating inferred specifications into shared libraries, producing formalized interfaces portable across microservices. These specifications encode method behavior as preconditions and postconditions, enabling compatibility checks and early detection of integration issues even without source access. Automated reasoning tools can also validate interactions between services by ensuring postconditions align with caller assumptions.

Empirically, this approach increased test generation by 35–193\% and raised mutation coverage above 94\% across key method categories. Overall, our scalable, language-agnostic framework grounded in formal semantics advances automated verification and improves test coverage, reliability, and CI/CD integration for distributed software systems.

\section{Conclusion}

As modern software systems grow increasingly distributed and modular, the need for tools that support long-term evolution and maintainability becomes critical. This paper presents a static analysis tool that automatically infers formal specifications from Java methods using weakest precondition and strongest postcondition analysis. By embedding these specifications into shared libraries, the tool provides a lightweight and scalable mechanism for verifying interface contracts, generating comprehensive test suites, and improving software reliability — all without requiring access to the full implementation code.

Our evaluation shows that inferred specifications significantly enhance mutation testing outcomes and test completeness across a diverse set of method categories. These improvements are particularly pronounced in categories involving control flow, factory delegation, and state modification, where traditional unit testing often falls short. By enabling automated validation of behavioral contracts at service boundaries, the tool helps detect integration bugs earlier and reduces long-term maintenance costs.

This is important because changes in shared libraries or service interfaces can easily introduce defects that manifest only during deployment. Early detection of such issues prevents costly downstream errors, improves developer confidence when modifying or extending code, and ensures that services interact correctly in complex or distributed systems. In turn, this leads to more reliable software, shorter development cycles, and reduced effort for regression testing and maintenance.

Beyond testing, the inferred specifications serve as durable, machine-checkable documentation that can guide developers during refactoring, onboarding, and evolution. This directly supports key activities in software maintenance, especially in microservice architectures where teams and services evolve independently.

In contrast to manual specification or natural-language approaches, our method requires no annotations or structured comments, making it highly suitable for industrial contexts with large, legacy codebases. The integration with mutation testing frameworks such as PiTest further demonstrates the tool’s compatibility with modern CI/CD pipelines.

While this paper focuses on specification inference and validation, future work will examine how these inferred contracts can be leveraged to improve automated testing of library clients and service integrations. This will include investigating their role in CI/CD pipelines, evaluating their effectiveness in cross-service verification, and conducting user studies to measure their impact on developer comprehension and maintenance workflows. Additional research will also explore optimizing the tool for large-scale performance, refining inference heuristics for dynamic and side-effect-heavy methods, and broadening support for other programming languages and ecosystems.

Overall, this work demonstrates that formal specification inference — when automated and integrated into development workflows and shared libraries — is both feasible and modular. By leveraging static analysis, symbolic execution, and heuristic techniques, the approach can generate specifications across large codebases without requiring manual annotation. These inferred specifications improve testability by guiding automated unit test generation, enhance reliability by detecting integration bugs and validating interface contracts, and support maintainability by providing machine-checkable documentation that captures expected behavior even as software evolves. Consequently, automated specification inference offers tangible benefits for developers working on complex, modular, or distributed systems, where ensuring correctness and reducing regression risk is particularly challenging.

\section*{Data Availability Statement}

The data supporting the findings of this study are not publicly available due to reproducibility constraints. The experimental results depend on outputs generated by large language models (LLMs), whose behavior may vary across model versions or updates. As such, identical reproduction of the AI-generated data cannot be guaranteed. However, all analysis procedures and configurations used in this study are available from the authors upon reasonable request.

\input{references}

\end{document}
